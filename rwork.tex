\section{\textcolor{red}{Related work}}
\label{sec:rw}


Existing methods in analyzing and characterizing FCNs rely heavily on graph analysis measures ~\cite{10.3389/fncom.2014.00051, TERMENON2016172, Aurich2015, ANDELLINI2015183} such as clustering coefficient and node degree. These measures are well known to summarize a single weighted network and also to compare FCNs within a collection of networks. For network comparison, typically, there are two cases: comparing two individual networks or comparing two collections of networks where the networks may be paired or unpaired between collections. Networks can be compared either at single nodes and links ~\cite{10.3389/fncom.2014.00051}, \cite{7270846}, or via some functional or transform summarizing each network~\cite{Simpson2013} using graph analysis approaches. Graph analysis measures, however, have been criticized for being dependent on the choice of brain parcellation ~\cite{Hilgetag2016, 10.3389/fnhum.2016.00096}, and network link threshold ~\cite{GARRISON2015651}. Graph analysis methods typically use binary measures of link strength, ignoring the weights of links. However, not all FCN links are equal. Some links are expected to be detected more frequently or be more strongly detected than others, and recognizing these differences in link strength allows us to understand the function within networks. Weighted graph analysis methods have been proposed ~\cite{RUBINOV20101059}, but are still susceptible to variation with respect to network density~\cite{10.3389/fncom.2014.00051}. Graph analysis of weighted brain networks at multiple thresholds has also been proposed~\cite{DRAKESMITH2015313}.
%An example of a traditional graph analysis approach toward analyzing FCNs is shown in Figure~\ref{fig:rips}.


Over the past few years, multi-site fMRI studies have become more popular due to their ability to quickly recruit participants and achieve larger sample sizes, which leads to increased statistical power~\cite{noble2017multisite}. This approach can help researchers studying uncommon diseases or diverse populations~\cite{dansereau2017statistical}. Additionally, multi-site fMRI studies can help to identify and control for potential confounds, such as differences in participant demographics. However, using different scanners and different imaging parameters can introduce non-biological variability, may diminish the statistical power and provide false findings. This is a well-known issue in fMRI research, and it is important to carefully consider these factors when designing and conducting fMRI studies~\cite{shinohara2017volumetric}.

%
% While some studies have reported site or scanner effects in fMRI data, only a few have attempted to standardize protocols and image acquisition parameters to mitigate these effects~\cite{chavez2018novel,shinohara2017volumetric}.

% Despite efforts to standardize protocols and image acquisition parameters, scanner-to-scanner variation due to the use of scanners from different manufacturers still exists~\cite{noble2017multisite}. An approach based on independent component analysis (ICA) has been used in one study to reduce scanner differences in multi-site resting-state fMRI post-acquisition~\cite{feis2015ica}. However, this approach did not entirely eliminate the structured noise that arises from the use of different scanners~\cite{yu2018statistical}. As a result, the development of harmonization techniques for multi-site data has become a growing topic in neuroimaging~\cite{roffet2022assessing}.

% One of the prominent and fast harmonization methods available to reduce multi-site or multi-scanner effects is ComBat~\cite{ingalhalikar2021functional,fortin2017harmonization,fortin2018harmonization,bell2022harmonization,yu2018statistical}. However, existing literature suggests that ComBat may not fully retain inter-subject biological variability following harmonization, especially in the presence of non-linear scanner contributions~\cite{Cetin-Karayumak2020.11.20.390120,2023HarmonizedDM}. While this was demonstrated for diffusion data, the principles are equally applicable to fMRI data.

%
TDA of networks goes beyond graph-theoretic analysis by utilizing tools from computational topology to describe the architecture of networks or data structures in more flexible ways ~\cite{Topology_and_data, ghrist2008barcodes}. In particular, it encodes higher order (not just pairwise) interactions in the system and studies topological features of a network across all possible thresholds. Persistent homology (PH), an advanced technique in TDA, is an emerging tool in studying complex networks, including brain networks ~\cite{5872535, 7164127, 10.1371/journal.pcbi.1002581}. PH-based methods have shown promising results in modeling transitions between brain states in fMRI data~\cite{Saggar2018}. There are many excellent introductions to PH, such as the books ~\cite{ghrist2014elementary, oudot2015persistence, zomorodian2005topology} and the papers ~\cite{patania2017topological, Topology_and_data, edelsbrunner2008surveys, ghrist2008barcodes, weinberger2011persistent}. Existing literature proposes using persistent homology to analyze functional connectivity networks (FCNs) from resting state fMRI data~\cite{kumar2023robustness}. The study presents a pipeline using topological data analysis techniques demonstrating FCN metrics are statistically similar across varied sampling periods. This suggests persistent homology provides a robust topological representation of FCNs invariant to acquisition parameters, potentially removing noise in multi-site studies and improving group comparison effect sizes. 
%An example of applying PH (topology-based analysis) on an FCN is shown in Figure~\ref{fig:PH}.