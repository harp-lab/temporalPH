\section{\textcolor{red}{Discussion}}
\label{sec:discussion}

MRI scanners across the world have different configurations and field strengths~\footnote{For example, the MRI machines available for research in the state of Alabama in the United States - Siemens 7T Magnetom and 3T Verio at Auburn University, Siemens 3T Prisma at the University of Alabama  Birmingham, Philips 3T Achieva at the University of South Alabama and Siemens 3T Prisma at the University of Alabama Tuscaloosa - all have different configurations}. Data acquired from different scanners with different parameters have some degree of noise in them due to non-neural variability introduced by different scanner configurations and data acquisition parameters, which makes it difficult for datasets acquired from different machines to be pooled into one large dataset for analysis within a single framework. As a result, conducting research on brain networks obtained from fMRI is primarily concentrated at localized sites. Such studies are limited by the fact that the number of subjects that can be scanned at a single scanner is limited, which tends to reduce the sample size and hence the generalizability of the results. This can be overcome by conducting multi-site studies with sites that are geographically closer to the population of interest. However, as mentioned above, data acquired from different scanners and parameters introduces noise, which reduces the effect of interest that is neural in origin, and hence reduces the utility of such a multi-site effort. Our paper aims to address precisely this aspect. Accordingly, we have proposed persistent homology-based techniques (which are based on TDA) as a means to overcome noise introduced by non-neural factors such as different data acquisition parameters and extract the inherent structural topology underlying brain networks characterized by fMRI. Computational topology is known to extract underlying shapes from complex data structures. We have shown that the topology-based metric for the brain network is invariant to data acquisition parameters such as sampling period. The metric captures the underlying shape of the brain network in topological space, thus creating a common ground to facilitate multi-site data-driven analysis of fMRI datasets (acquired from different scanners).

We were able to establish the efficacy of the TDA-based pipeline by comparing it against the traditional data analysis pipeline presented in Section III.D.3. In a traditional pipeline, raw FCNs are used directly instead of using a topological feature. The results of this pipeline are a strong indication that these traditional techniques are not able to establish similarity across FCNs of the same subjects acquired with different TRs and acquisition parameters. While the opposite was true for the TDA-based metric, wherein we showed both qualitatively and quantitatively that the metric remains statistically invariant across same subjects irrespective of the sampling period with which resting state fMRI data was acquired.
This demonstrates the utility of TDA-based analysis because, in principle, data acquired using different parameters from the same subject should still capture the same brain network. Certain limitations of our work must be kept in mind while interpreting the results. Even though the sampling period was different across the three different acquisitions, some other parameters such as number of volumes, multiband factor, FOV and voxel size also varied across the three cohorts. Ideally, we would want to investigate the limits of parameter variability that would show invariance in the TDA metric by varying only one parameter at a time. This will be part of our future work in this area. Also, future work can investigate whether the TDA metric is invariant to differences in other variables such as vendors (such as Siemens, GE and Philips) and scanner models within a given vendor.



